\documentclass{article}
\usepackage{polski}
\usepackage[utf8]{inputenc}
\usepackage{amsmath}

\begin{document}
\section{Algorytm pszczeli}
Idea algorytmu została oparta na poszukiwaniu pożywienia przez kolonie pszczół. Pszczoły w procesie
poszukiwania pożywienia wykształciły różne techniki(np. ”taniec pszczół”
- ang. waggle dance) usprawniające komunikację z pozostałymi pszczołami
na temat lokalizacji źródeł pożywienia jak i techniki określania najlepszych
dostępnych miejsc pokarmu.

\subsection{Taniec pszczeli}

Kiedy pszczoła, zwana skautem, lokalizuje bogate źródło pożywienia, pobiera małą próbkę pożywienia i wraca do ula, aby powiadomić pozostałe
pszczoły o dostępnym dobrej, jakości nektarze. Wykonuje wówczas tak zwany taniec pszczeli. Taniec pszczół przekazuje trzy podstawowe informacje o znalezionym pożywieniu:
\begin{itemize}
\item odległość od ula do źródła,
\item kierunek(kąt pomiędzy słońcem a źródłem pożywienia),
\item jakość nektaru, która znajduje się w danym źródle\\
\end{itemize}

\subsection{Podział pszczół}
\begin{itemize}
\item Skauci - pszczoły, które nie mają żadnych informacji gdzie może znajdować się pokarm, poszukują jedzenia w sposób losowy.
\item Rekruci - pszczoły, które obserwują ”taniec pszczół” wykonywany przez inne pszczoły i wybierają odpowiednie źródło pokarmu na podstawie informacji przekazanych przez taniec.\\
\end{itemize}


\subsection{Parametry algorytmu}


\begin{itemize}\itemsep2pt
 \item[$n$] -- liczba pszczół
 \item[$m$] -- liczba wybranych miejsc spośród \emph{n}  odwiedzonych
 \item[$e$] -- liczba najlepszych miejsc sprośród \emph{m}  wybranych
 \item[$nep$] -- liczba pszczół rekrutowanych do \emph{e}  najlepszych miejsc 
 \item[$nsp$] -- liczba pszczół rekrutowana do \emph{m-e}  gorszych miejsc
 \item[$ngh$] -- rozmiar sąsiedztwa\\ 
\end{itemize}

\subsection{Schemat działania algorytmu} 
\begin{description}
 \item[Krok 1] Algorytm inicjalizuje całą (\emph{n}) populację losowymi wartościami.
 \item[Krok 2] Algorytm oblicza funkcję celu.
 \item[Krok 3] Algorytm oblicza rozmiar sąsiedztwa równy \begin{math}
                                                       rozmiar\_problemu \cdot ngh
                                                      \end{math}
 \item[Krok 4] Tworzenie nowej populacji:
\begin{description}
 \item [Krok 4.1] Algorytm wybiera \emph{m} miejsc do szukania w sąsiedztwie
 \item [Krok 4.3] Pszczoły (\emph{nsp})  zostają rekrutowane do wybranych miejsc (więcej pszczół (\emph{nep}) do \emph{e} najlepszych miejsc)
 \item [Krok 4.4] Algorytm wybiera dla każdego przesukiwanego miejsca najlepszą pszczołę na podstawie funkcji celu.
 \item [Krok 4.5] Algorytm przydziela pozostałe (\emph{n-m}) pszczoły do szukania w nowych losowych miejscach.\\
\end{description}


\end{description}


\subsection{Wybór rozwiązania}
W każdej iteracji cała populacja pszczół jest sortowana po wartości funkcji celu. Wybierane jest rozwiązanie z minimalnym kosztem i rozpoczyna się kolejna iteracja.





\end{document}
